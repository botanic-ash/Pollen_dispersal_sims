% Options for packages loaded elsewhere
\PassOptionsToPackage{unicode}{hyperref}
\PassOptionsToPackage{hyphens}{url}
%
\documentclass[
]{article}
\usepackage{lmodern}
\usepackage{amssymb,amsmath}
\usepackage{ifxetex,ifluatex}
\ifnum 0\ifxetex 1\fi\ifluatex 1\fi=0 % if pdftex
  \usepackage[T1]{fontenc}
  \usepackage[utf8]{inputenc}
  \usepackage{textcomp} % provide euro and other symbols
\else % if luatex or xetex
  \usepackage{unicode-math}
  \defaultfontfeatures{Scale=MatchLowercase}
  \defaultfontfeatures[\rmfamily]{Ligatures=TeX,Scale=1}
\fi
% Use upquote if available, for straight quotes in verbatim environments
\IfFileExists{upquote.sty}{\usepackage{upquote}}{}
\IfFileExists{microtype.sty}{% use microtype if available
  \usepackage[]{microtype}
  \UseMicrotypeSet[protrusion]{basicmath} % disable protrusion for tt fonts
}{}
\makeatletter
\@ifundefined{KOMAClassName}{% if non-KOMA class
  \IfFileExists{parskip.sty}{%
    \usepackage{parskip}
  }{% else
    \setlength{\parindent}{0pt}
    \setlength{\parskip}{6pt plus 2pt minus 1pt}}
}{% if KOMA class
  \KOMAoptions{parskip=half}}
\makeatother
\usepackage{xcolor}
\IfFileExists{xurl.sty}{\usepackage{xurl}}{} % add URL line breaks if available
\IfFileExists{bookmark.sty}{\usepackage{bookmark}}{\usepackage{hyperref}}
\hypersetup{
  hidelinks,
  pdfcreator={LaTeX via pandoc}}
\urlstyle{same} % disable monospaced font for URLs
\usepackage[margin=1in]{geometry}
\usepackage{graphicx,grffile}
\makeatletter
\def\maxwidth{\ifdim\Gin@nat@width>\linewidth\linewidth\else\Gin@nat@width\fi}
\def\maxheight{\ifdim\Gin@nat@height>\textheight\textheight\else\Gin@nat@height\fi}
\makeatother
% Scale images if necessary, so that they will not overflow the page
% margins by default, and it is still possible to overwrite the defaults
% using explicit options in \includegraphics[width, height, ...]{}
\setkeys{Gin}{width=\maxwidth,height=\maxheight,keepaspectratio}
% Set default figure placement to htbp
\makeatletter
\def\fps@figure{htbp}
\makeatother
\setlength{\emergencystretch}{3em} % prevent overfull lines
\providecommand{\tightlist}{%
  \setlength{\itemsep}{0pt}\setlength{\parskip}{0pt}}
\setcounter{secnumdepth}{-\maxdimen} % remove section numbering

\author{}
\date{\vspace{-2.5em}}

\begin{document}

\#' --- \#' output: github\_document \#' ---

\#' Exploratory data analysis for one population simulation of ideal
sampling scenarios

\#' libraries library(dplyr) library(tidyr) library(ggplot2)
theme\_set(theme\_bw())

\#load in data load(``alleles\_capt\_ideal\_onepop.Rdata'')

\#' GET DATA IN PROPER FORMAT FOR PLOTTING AN ANALYSIS \#' We need all
the data which is currently in 3 separate 3D matrices to be combined
into one large matrix in tidy format

\#' 1. Converting 3D results matrices to 2D for plotting and analysis
purposes same\_long=NULL eligible\_long=NULL skewed\_long=NULL j=1 for(i
in 1:50) \{ temp = rbind(prop\_capt\_all\_same{[},,j{]},
prop\_capt\_all\_same{[},,(j+1){]}) same\_long = rbind(same\_long, temp)
j = j+2 \} j=1 for(i in 1:50) \{ temp =
rbind(prop\_capt\_all\_eligible{[},,j{]},
prop\_capt\_all\_eligible{[},,(j+1){]}) eligible\_long =
rbind(eligible\_long, temp) j = j+2 \} j=1 for(i in 1:50) \{ temp =
rbind(prop\_capt\_skewed{[},,j{]}, prop\_capt\_skewed{[},,(j+1){]})
skewed\_long = rbind(skewed\_long, temp) j = j+2 \} rm(temp)

\#' 2. Binding all data frames together into one huge data frame \#'
This is considered tidy format? tidy\_df = rbind(same\_long,
eligible\_long, skewed\_long) tidy\_df = as.data.frame(tidy\_df) \#'
Just adding a column indicating the number of failures (number of
alleles not captured) for binomial data
tidy\_df\(num_fails = c(as.numeric(tidy_df\)total\_alleles) -
as.numeric(tidy\_df\$num\_capt))

\#EXPLORATORY DATA ANALYSIS \#1. Examine the data View(tidy\_df)

\#2. Visualizing the data \#Plot of total alleles present in simulation
replicates to see the distribution of variation between replicates
\#Each bar represents a different count of alleles \#The 3 increments on
the plot show the allele count being present in one, two, or three
different simulations by chance \#However, most replicates model a
different number of total alleles due to the stochastic simulations
ggplot(tidy\_df, aes(x=as.numeric(total\_alleles))) + geom\_bar() +
theme(axis.text.x=element\_blank(), axis.ticks.x=element\_blank(),
axis.text.y=element\_blank(), axis.ticks.y=element\_blank())

\#Showing a table of all instances of total alleles present in
simulations \#Occurence values of 2805 = present in 1 independent
simulation (935 scenarios x pollen donors = 1 simulation) \#Occurence
values of 5610 = present in 2 independent simulations \#Occurence values
of 8415 = present in 3 independent simulations \#Also, the total alleles
simulated in simulations ranges from 235 to 288 \#--this results in
variation during sampling as well, for the same sample size of 50 seeds,
\#you may capture more diversity in the simulation with 235 alleles than
288 total alleles table(tidy\_df\$total\_alleles)

\#Plotting all the data! \#Plot proportion of alleles captured vs total
number of seeds sampled for each number of maternal trees \#Here we see
that in scenarios with fewer maternal trees sampled (facet 1, 2), there
are greater differences in the proportion of alleles captured between
pollen donor scenarios (large difference between the curves) \#In
scenarios with more maternal trees sampled (facet 50, 100), similar
proportion of alleles are captured across donor types (though there is
less information here with fewer scenarios). \#Additionally, comparing a
given pollen donor type across varying number of trees sampled (compare
a color across facets), we see that when more maternal trees are
sampled, \#much more diversity is captured (curve gets higher--greater
proportion of alleles captured ). \#Lastly, when more seeds are sampled
per tree (going along the x-axis), we see a slight increase in the
diversity captured (see curves upward) tidy\_df \%\textgreater\%
ggplot(aes(x=as.numeric(total\_seeds), y=as.numeric(prop\_capt),
color=donor\_type)) + geom\_point(alpha=0.25) +
facet\_wrap(vars(maternal\_trees)) + ylim(0,1) +
theme(axis.text.x=element\_blank(), axis.ticks.x=element\_blank(),
axis.text.y=element\_blank(), axis.ticks.y=element\_blank())

\#Inspecting scenarios more closely--100 seeds total sampled, faceted by
number of maternal trees sampled again and proportion of alleles
captured on the y-axis (plotted with jitter to spread points across
x-axis) \#Again, we see the trend of pollen donor type appearing to have
more influence on the proportion of alleles captured in scenarios with
fewer maternal trees sampled, since most of the diversity would be
coming from pollen donors in these cases. tidy\_df \%\textgreater\%
filter(total\_seeds==100) \%\textgreater\%
ggplot(aes(x=as.numeric(total\_seeds), y=as.numeric(prop\_capt),
color=donor\_type)) + geom\_point(alpha=0.25) +
facet\_wrap(vars(maternal\_trees)) + ylim(0,1) + geom\_jitter() +
theme(axis.text.x=element\_blank(), axis.ticks.x=element\_blank(),
axis.text.y=element\_blank(), axis.ticks.y=element\_blank())

\#Boxplots of specific scenarios for each \# of maternal tree sampled,
100 total seeds sampled \#Again, this shows the variation in alleles
captured for each pollen donor type in scenarios with fewer maternal
trees \#When many maternal trees are sampled, there is no difference in
diversity captured tidy\_df \%\textgreater\% filter(total\_seeds==100)
\%\textgreater\% ggplot(aes(x=as.numeric(total\_seeds),
y=as.numeric(prop\_capt), color=donor\_type)) +
geom\_boxplot(alpha=0.25) + facet\_wrap(vars(maternal\_trees)) +
ylim(0,1) + theme(axis.text.x=element\_blank(),
axis.ticks.x=element\_blank(), axis.text.y=element\_blank(),
axis.ticks.y=element\_blank())

\#Plotting a scenario of 1 maternal tree sampled to better view the
curve of the data tidy\_df \%\textgreater\% filter(maternal\_trees==1)
\%\textgreater\% ggplot(aes(x=as.numeric(total\_seeds),
y=as.numeric(prop\_capt), color=donor\_type)) + geom\_point(alpha=0.25)
+ facet\_wrap(vars(maternal\_trees)) + ylim(0,1) +
theme(axis.text.x=element\_blank(), axis.ticks.x=element\_blank(),
axis.text.y=element\_blank(), axis.ticks.y=element\_blank())

\end{document}
